\documentclass[11pt,a4paper]{moderncv}

\moderncvtheme[grey]{classic}
\usepackage[utf8]{inputenc}
\usepackage[scale=0.7]{geometry}

\usepackage[unicode]{hyperref}
\definecolor{linkcolour}{rgb}{0,0.2,0.6}
\hypersetup{colorlinks,breaklinks,urlcolor=linkcolour, linkcolor=linkcolour}

\firstname{Vyacheslav}
\familyname{Barinov}
\address{}{Moscow, Russia}
% \mobile{+7 ...}
\email{rayslava@gmail.com}
\homepage{https://github.com/rayslava}
\extrainfo{Software should be beautiful. Both inside and outside.}

\makeatletter
\renewcommand*{\bibliographyitemlabel}{\@biblabel{\arabic{enumiv}}}
\makeatother

\begin{document}
\maketitle

\section{Brief description}
\cvitem
  {}
  {My main direction - C/C++ Linux development.\newline{}
  This symbiosis allows me to create interacting system of a mobile client and a web service.\newline{}
  I really enjoy OpenSource software and do my best to develop and contribute OpenSource projects which I'm interested in.}

\section{OpenSource Projects}
\subsection{C++ / Qt}
\cvitem
  {Leechcraft}
  {\url{https://github.com/rayslava/leechcraft}\newline{}
  A modular internet client. My part is Twitter client plugin}
\subsection{C++}
\cvitem
  {Massacre}
  {\url{https://github.com/rayslava/Massacre}\newline{}
  A simple 3D engine}

\section{Commercial Projects}
  \subsection{C/C++ / Linux}
  \cvitem
    {Tizen}
    {\url{http://tizen.org}\newline{}
    Toolchain support}

\section{Experience}
\cventry
  {Feb 2012 - still here}
  {Software Engineer}
  {Samsung Research Russia, Moscow}
  {}{}
  {Toolchain support (GCC compatibility issues maintaining and bugfixes)\newline{}
Solving architecture related issues (porting from arm to arm64)\newline{}
Native mobile software optimization (C/C++)\newline{}
Profiling and optimization tool development (Linux kernel module profiling part, python GUI and statistical and reporting module)\newline{}
Software development.}
\cventry
  {Sep 2006 - Feb 2012}
  {Expert Engineer}
  {InterRadioService LLC. Moscow, Russia}
  {}{}
  {Hardware engineering\newline{}
Software development (mostly for hardware support)\newline{}
Maintaining compatibility features for ancient (40 yrs and more) devices.\newline{}
Fieldwork with hardware and support during full-scale tests\newline{}}

\section{Development skills}
\subsection{Programming Languages}
\cvitem
  {Expert}{C, C++}
\cvitem
  {Intermediate}{Python, Pascal, Common Lisp, FORTH, x86 Assembly Language}
\cvitem
  {Basic}{Perl, Scala, ARM Assembly Language}
\subsection{Frameworks and tools}
\cvitem{Development}{Build system related tools: Makefile, m4, cmake, shell scripting}
\cvitem{Debug}{Low-level system debug: linux kdb, U-Boot debug}
\cvitem{GUI}{Qt (C++ and Python versions)}

\section{Server administration skills}
\cvitem
  {Jenkins}{Deployment and support on Linux}
\cvitem
  {OBS}{Setup, configuration and bootstrapping from scratch for rpm-based Tizen OS}
\cvitem
  {General}{Setting up and maintaining ``LAMP'' nodes — \url{http://rayslava.com} for example.}

\section{Computer skills}
  \cvitem
  {OS}{Linux (Gentoo, Arch)}
  \cvitem
  {VCS}{git, svn}

\section{Education}
  \cventry
    {2005 - 2011}
    {Electronics Engineering}
    {Bauman's Moscow State Technical University}
    {}{}
    {M.Sc. in Electronics device and systems engineering}

\section{Public activity}
  \cvitem{LinkedIn}{\url{http://www.linkedin.com/pub/alexey-denisov/2a/bab/a29}}
  \cvitem{GitHub}{\url{https://github.com/rayslava}}

\end{document}

%%% Local Variables: 
%%% mode: latex
%%% TeX-master: t
%%% End: 
