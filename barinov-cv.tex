\documentclass[11pt,a4paper]{moderncv}
\moderncvtheme[grey]{classic}
\usepackage[utf8]{inputenc}
\usepackage[scale=0.9]{geometry}
\definecolor{linkcol}{rgb}{0,0.2,0.6}
\usepackage[unicode,colorlinks,breaklinks,urlcolor=linkcol,linkcolor=linkcol]{hyperref}


\firstname{Viacheslav}
\familyname{Barinov}
\address{Moscow, Russia}{}
% \mobile{+7 ...}
\email{rayslava@gmail.com}
\homepage{rayslava.com}
\photo[92pt][0.4pt]{photo.jpg}

\makeatletter
\renewcommand*{\bibliographyitemlabel}{\@biblabel{\arabic{enumiv}}}
\makeatother

\begin{document}
\maketitle

\section{Brief description}
\cvitem {} {My major activity -- C/C++ development on Linux-based platforms.\newline{} I really enjoy the OpenSource
  software and interaction with community and do my best to contribute to OpenSource projects which I'm interested in or
  have to deal with at day-time job.}
\section{Commercial Projects}
  \subsection{C/C++ / Linux}
  \cvitem {Tizen}
  {\url{https://tizen.org}\newline{}
    Tizen toolchain support and enhancement.\newline{}
    Our team is working on toolchain support and enhancement accordingly to requests of Samsung developers all over the
    world and own vision of perfect toolchain. We are the guys who created an ARM64~Tizen~3.0 in the beginning of 2014
    and stabilized the OS even before first 64bit Android release.\newline{}
    Digging inside GCC and Binutils to find why stack trace tells lies to developers, debugging on-device soft using gdb
    through port forwarded using netcat over Wi-Fi because kernel is not finished and there is no other way to connect,
    finding a way to implement needed feature the right way instead of workaround in the limited time--that's about me.
    \newline{}
    After promotion to project lead the task set widened with looking for appropriate customer for such experience
    and directing team members' fun into scope needed by commercial divisions. Six years of successful projects say
    I'm not totally useless in this field as well.\newline{}
    Our team was invoked into code analysis movement and I joined dynamic analysis forces to improve and promote sanitizers
    among Samsung. Now, as far as I know, we're the first team who performed full-system sanitization (not sure if someone
    tried that anyway).
  }
\section{Experience}
\cventry
  {2021 - still here}
  {Staff Engineer}
  {Samsung Research Russia}
  {Moscow}{Russia}
  {Rust evangelism among SRR developers, research projects}
\cventry
  {2017 - 2021}
  {Staff Engineer/Team Lead}
  {Samsung Research Russia}
  {Moscow}{Russia}
  {Toolchain improvement, dynamic software analysis research, Improving Tizen
    code base from toolchain point of view: dynamic analysis (sanitizers), new
    optimizations. Applying dynamic binary analysis technologies to embedded
    systems (\href{https://github.com/Samsung/TizenRT}{Tizen~RT})}
\cventry
  {2014 - 2017}
  {Leading Engineer/Team Lead}
  {Samsung Research Russia}
  {Moscow}{Russia}
  {Toolchain support and enhancement. Improving Tizen code base from toolchain point of view: dynamic analysis
    (sanitizers), new optimizations}
\cventry
  {2013 - 2014}
  {Leading Engineer/Team Lead}
  {Samsung Research Russia}
  {Moscow}{Russia}
  {Toolchain support (GCC compatibility issues maintaining and bugfixes)\\
Solving architecture related issues (porting from arm to arm64)\\}
\cventry
  {Feb 2012 - 2013}
  {Software Engineer}
  {Samsung Research Russia}
  {Moscow}{Russia}
  {Native mobile software optimization (C/C++)\newline{}
Profiling and optimization tool development (Linux kernel module profiling part, python GUI and statistical and reporting module)\newline{}
Software development.}
\cventry
  {Sep 2006 - Feb 2012}
  {Expert Engineer}
  {InterRadioService LLC.}
  {Moscow}{Russia}
  {Hardware engineering\newline{}
Software development (mostly for hardware support).\newline{}
Maintaining compatibility features for ancient (40 yrs and more) devices.\newline{}
Fieldwork with hardware and support during full-scale tests.\newline{}}

\section{Development skills}
\subsection{Programming Languages}
\cvitem
  {Expert}{C, C++}
\cvitem
  {Intermediate}{Rust, Python, ARM Assembly Language}
\cvitem
  {Basic}{Perl, x86 Assembly Language}

\subsection{Frameworks and tools}
\cvitem{Development} {Build system related tools: Makefile, m4, cmake, shell scripting, the GNU tool set built around
  traditional *nix software development.}

\cvitem{Debug}{Low-level system debug: gdb, kdb, U-Boot debug}

\cvitem{GUI}{Qt (C++ and Python versions), basic web-interface}

\cvitem{OS}{Linux-based (from LFS and Gentoo to full rpm-based Tizen)}

\cvitem{VCS}{git, svn}

\section{Server administration skills}
\cvitem {OBS}{Deep knowledge of service internals, have experience in setup, configuration and bootstrapping from
  scratch for rpm-based Tizen OS}

\cvitem {General}{Setting up and maintaining all servers needed for work -- CI, source hosting, web-interfaces just need enough time.}
\section{Publications}
\subsection{Software Analysis}
  \cventry
    {2019}
    {GNU~Tools~Cauldron}
    {\url{https://gcc.gnu.org/wiki/cauldron2019}}
    {}{}
    {Report \textit{Annotating \texttt{std::string} with AddressSanitizer}}
  \cventry
    {2017}
    {GNU~Tools~Cauldron}
    {\url{https://gcc.gnu.org/wiki/cauldron2017}}
    {}{}
    {Report \textit{Applying GNU GCC Address Sanitizer to whole Linux distribution}}
  \cventry
    {2017}
    {2017 Ivannikov ISPRAS Open Conference}
    {\url{http://isprasopen.ru/en/}}
    {}{}
    {Report \textit{Applying GCC-Based Address Sanitizer Dynamic Analysis
        Technology to Tizen OS }.\newline{}
      Published in Proceedings of 2017 Ivannikov ISPRAS Open Conference
      (ISPRAS).\newline{}
      DOI:~\href{https://doi.org/10.1109/ISPRAS.2017.00015}{10.1109/ISPRAS.2017.00015}.
    }
\section{Education}
  \cventry
    {2015}
    {People Management}
    {\url{http://www.luxoft-training.com/}}
    {Moscow}{Russia}
    {Project manager tasks related to grouping people into team}
  \cventry
    {2014}
    {Project Management}
    {\url{http://www.luxoft-training.com/}}
    {Moscow}{Russia}
    {Project manager quickstart}
  \cventry
    {2005 - 2011}
    {Electronics Engineering}
    {Bauman's Moscow State Technical University}
    {}{}
    {M.Sc. in Electronics device and systems engineering}
\section{Public activity}
  \cvitem{LinkedIn}{\url{https://linkedin.com/in/rayslava}}
  \cvitem{GitHub}{\url{https://github.com/rayslava}}
\end{document}

%%% Local Variables:
%%% mode: latex
%%% TeX-master: t
%%% End:
