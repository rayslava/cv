\documentclass[11pt,a4paper]{moderncv}

\moderncvtheme[grey]{classic}
\usepackage[utf8]{inputenc}
\usepackage[scale=0.9]{geometry}

\usepackage[unicode]{hyperref}
\definecolor{linkcolour}{rgb}{0,0.2,0.6}
\hypersetup{colorlinks,breaklinks,urlcolor=linkcolour, linkcolor=linkcolour}

\firstname{Vyacheslav}
\familyname{Barinov}
\address{Moscow, Russia}{}
% \mobile{+7 ...}
\email{rayslava@gmail.com}
\homepage{http://rayslava.com}
\photo[64pt][0.4pt]{photo.jpg}

\makeatletter
\renewcommand*{\bibliographyitemlabel}{\@biblabel{\arabic{enumiv}}}
\makeatother

\begin{document}
\maketitle

\section{Brief description}
\cvitem {} {My major activity -- C/C++ development on Linux-based platforms.\newline{} I really enjoy the OpenSource
  software and interaction with community and do my best to contribute to OpenSource projects which I'm interested in or
  have to deal with at day-time job.}
\section{Commercial Projects}
  \subsection{C/C++ / Linux}
  \cvitem
    {Tizen}
    {\url{https://tizen.org}\newline{}
      Tizen toolchain support and enhancement.\newline{} Our team is working on toolchain support by requests of
      developers all over the world and enhancement. We are the guys who created an ARM64~Tizen~3.0 in 2014 beginning
      and stabilized the OS during the year even before Android release.}
\section{Experience}
\cventry
  {2014 - still here}
  {Team Lead/PM}
  {Samsung Research Russia}
  {Moscow}{Russia}
  {Toolchain support and enhancement. Improving Tizen code base from toolchain point of view: dynamic analysis
    (sanitizers), new optimizations}
\cventry
  {2013 - 2014}
  {Leading Engineer/Team Lead}
  {Samsung Research Russia, Moscow}
  {}{}
  {Toolchain support (GCC compatibility issues maintaining and bugfixes)\\
Solving architecture related issues (porting from arm to arm64)\\}
\cventry
  {Feb 2012 - 2013}
  {Software Engineer}
  {Samsung Research Russia, Moscow}
  {}{}
  {Native mobile software optimization (C/C++)\newline{}
Profiling and optimization tool development (Linux kernel module profiling part, python GUI and statistical and reporting module)\newline{}
Software development.}
\cventry
  {Sep 2006 - Feb 2012}
  {Expert Engineer}
  {InterRadioService LLC. Moscow, Russia}
  {}{}
  {Hardware engineering\newline{}
Software development (mostly for hardware support)\newline{}
Maintaining compatibility features for ancient (40 yrs and more) devices.\newline{}
Fieldwork with hardware and support during full-scale tests\newline{}}

\section{Development skills}
\subsection{Programming Languages}
\cvitem
  {Expert}{C, C++}
\cvitem
  {Intermediate}{ARM Assembly Language, Python, Pascal, Common Lisp, FORTH}
\cvitem
  {Basic}{Perl, Scala, x86 Assembly Language}

\subsection{Frameworks and tools}
\cvitem{Development}
{Build system related tools: Makefile, m4, cmake, shell scripting, all the tool set built around traditional *nix}
\cvitem{Debug}
{Low-level system debug: linux kdb, U-Boot debug}
\cvitem{GUI}
{Qt (C++ and Python versions), basic web-interface}

\section{Education}
  \cventry
    {2015}
    {People Management}
    {\url{http://www.luxoft-training.com/}}
    {}{}
    {Project manager}
  \cventry
    {2014}
    {Project Management}
    {\url{http://www.luxoft-training.com/}}
    {}{}
    {Project manager}
  \cventry
    {2005 - 2011}
    {Electronics Engineering}
    {Bauman's Moscow State Technical University}
    {}{}
    {M.Sc. in Electronics device and systems engineering}

\section{OpenSource Projects}
\subsection{C++ / Qt}
\cvitem
  {Leechcraft}
  {\url{https://github.com/rayslava/leechcraft}\newline{}
  A modular multipurpose internet client. My part is Twitter client plugin}

\subsection{C++}
\cvitem
  {Massacre}
  {\url{https://github.com/rayslava/Massacre}\newline{}
    A simple cross-platform 3D engine as standalone as it could be.}

\cvitem
  {chatsync}
  {\url{https://github.com/rayslava/chatsync}\newline{}
  Rather simple tool for synchronizing of chat rooms from different protocols}

\section{Server administration skills}
\cvitem
  {Jenkins}{Deployment and support on Linux}

\cvitem
  {OBS}{Deep knowledge of service internals, have experience in setup, configuration and bootstrapping from
    scratch for rpm-based Tizen OS}

\cvitem
  {General}{Setting up and maintaining \texttt{LAMP} nodes -- \url{http://rayslava.com} for example.}

\section{Computer skills}
  \cvitem
  {OS}{Linux (Gentoo, Arch)}
  \cvitem
  {VCS}{git, svn}

\section{Public activity}
  \cvitem{LinkedIn}{\url{https://linkedin.com/in/rayslava}}
  \cvitem{GitHub}{\url{https://github.com/rayslava}}

\end{document}

%%% Local Variables: 
%%% mode: latex
%%% TeX-master: t
%%% End: 
